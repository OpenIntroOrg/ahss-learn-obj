\documentclass[11pt]{article}
\usepackage[top=1.5cm,bottom=2cm,left=2cm,right= 2cm]{geometry}
\geometry{letterpaper}                   % ... or a4paper or a5paper or ... 
%\geometry{landscape}                % Activate for for rotated page geometry
\usepackage[parfill]{parskip}    % Activate to begin paragraphs with an empty line rather than an indent
\usepackage{graphicx}
\usepackage{amssymb}
\usepackage{epstopdf}
\usepackage{amsmath}            
\usepackage{multirow}    
\usepackage{multicol}    
\usepackage{changepage}
\usepackage{lscape}
\usepackage{enumitem}
\usepackage{ulem}
\DeclareGraphicsRule{.tif}{png}{.png}{`convert #1 `dirname #1`/`basename #1 .tif`.png}

\usepackage{xcolor}

\definecolor{oiB}{rgb}{.337,.608,.741}
\definecolor{oiR}{rgb}{.941,.318,.200}
\definecolor{oiG}{rgb}{.298,.447,.114}
\definecolor{oiY}{rgb}{.957,.863,0}

\definecolor{light}{rgb}{.337,.608,.741}
\definecolor{dark}{rgb}{.337,.608,.741}

\usepackage[colorlinks=false,pdfborder={0 0 0},urlcolor= dark,colorlinks=true,linkcolor=black]{hyperref}

\newcommand{\light}[1]{\textcolor{light}{\textbf{#1}}}
\newcommand{\dark}[1]{\textcolor{dark}{#1}}
\newcommand{\gray}[1]{\textcolor{gray}{#1}}


%\date{}                                           % Activate to display a given date or no date

%

\begin{document}

{\LARGE \textcolor{oiB}{Learning Objectives \hfill Chapter 5: Inference for numerical variables}} \\

%

\begin{enumerate}
\renewcommand\labelenumi{\textcolor{light}{\textbf{LO \theenumi.}}}

\item Use the $t$-distribution for inference on a single mean, difference of paired (dependent) means, and difference of independent means.

\item Explain why the $t$-distribution helps make up for the additional variability introduced by using $s$ (sample standard deviation) in calculation of the standard error, in place of $\sigma$ (population standard deviation).

\item Describe how the $t$-distribution is different from the normal distribution, and what �heavy tail� means in this context. 

\item Note that the $t$-distribution has a single parameter, degrees of freedom, and as the degrees of freedom increases this distribution approaches the normal distribution. 

\item Use a $t$-statistic, with degrees of freedom $df = n - 1$ for inference for a population mean:
\begin{itemize}
\item[-] Standard error: $SE = \frac{s}{\sqrt{n}}$
\item[-] Confidence interval: $\bar{x} \pm t_{df}^\star SE$
\item[-] Hypothesis test: $T_{df} = \frac{\bar{x} - \mu}{SE}$
\end{itemize}

\item Describe how to obtain a p-value for a $t$-test and a critical $t$-score ($t^\star_{df}$) for a confidence interval.

\end{enumerate}

\gray{
{\it
\vspace{-0.55cm}
\begin{itemize}
\renewcommand{\labelitemi}{{\textcolor{dark}{$\ast$}}}
\item Reading: Section 5.1 of OpenIntro Statistics
%\item Videos: To be posted
\item Test yourself:
\begin{enumerate}
\item What is the $t^\star$ for a 95\% confidence interval for a mean, where the sample size is 13.
\item What is the p-value for a hypothesis test where the alternative hypothesis is two-sided, the sample size is 20, and the test statistic, T, is calculated to be 1.75? 
\end{enumerate}
\end{itemize}
}}

%

\vspace{0.48cm}

%

\begin{enumerate}[resume]
\renewcommand\labelenumi{\textcolor{light}{\textbf{LO \theenumi.}}}

\item Define observations as paired if each observation in one dataset has a special correspondence or connection with exactly one observation in the other data set. 

\item Carry out inference for paired data by first subtracting the paired observations from each other, and then treating the set of differences as a new numerical variable on which to do inference (such as a confidence interval or hypothesis test for the average difference). 

\item Calculate the standard error of the difference between means of two paired (dependent) samples as $SE = \frac{s_{diff}}{\sqrt{n_{diff}}}$ and use this standard error in hypothesis testing and confidence intervals comparing means of paired (dependent) groups. 

\item Use a $t$-statistic, with degrees of freedom $df = n_{diff} - 1$ for inference for a population mean:
\begin{itemize}
\item[-] Standard error: $SE = \frac{s}{\sqrt{n}}$
\item[-] Confidence interval: $\bar{x}_{diff} \pm t_{df}^\star SE$
\item[-] Hypothesis test: $T_{df} = \frac{\bar{x}_{diff} - \mu_{diff}}{SE}$. Note that $\mu_{diff}$ is often 0, since often $H_0: \mu_{diff} = 0$.
\end{itemize}

\item Recognize that a good interpretation of a confidence interval for the difference between two parameters includes a comparative statement (mentioning which group has the larger parameter). 

\item Recognize that a confidence interval for the difference between two parameters that doesn�t include 0 is in agreement with a hypothesis test where the null hypothesis that sets the two parameters equal to each other is rejected. 

\end{enumerate}

\gray{
{\it
\vspace{-0.55cm}
\begin{itemize}
\renewcommand{\labelitemi}{{\textcolor{dark}{$\ast$}}}
\item Reading: Section 5.2 of OpenIntro Statistics
%\item Videos: To be posted
\item Test yourself:
\begin{enumerate}
\item 20 cardiac patients' blood pressure is measured before taking a medication, and after. For a given patient, are the before and after blood pressure measurements dependent (paired) or independent?
\item A random sample of 100 students were obtained and then randomly assigned into two equal sized groups. One group went on a roller coaster while the other in a simulator at an amusement park. Afterwards their blood pressure measurements were taken. Are the measurements dependent (paired) or independent?
\end{enumerate}
\end{itemize}
}}

%

\vspace{0.48cm}

%

\begin{enumerate}[resume]
\renewcommand\labelenumi{\textcolor{light}{\textbf{LO \theenumi.}}}

\item Calculate the standard error of the difference between means of two independent samples as $SE = \sqrt{\frac{s_1^2}{n_1} + \frac{s_2^2}{n_2}}$, and use this standard error in hypothesis testing and confidence intervals comparing means of independent groups.

\item Use a $t$-statistic, with degrees of freedom $df = min(n_1 - 1, n_2 - 1)$ for inference for a population mean:
\begin{itemize}
\item[-] Standard error: $\sqrt{\frac{s_1^2}{n_1} + \frac{s_2^2}{n_2}}$
\item[-] Confidence interval: $(\bar{x}_1 - \bar{x}_2) \pm t_{df}^\star SE$
\item[-] Hypothesis test: $T_{df} = \frac{(\bar{x}_1 - \bar{x}_2) - (\mu_1 - \mu_2)}{SE}$. Note that $\mu_{diff}$ is often 0, since often $H_0: \mu_1 - \mu_2 = 0$.
\end{itemize}

\end{enumerate}

\gray{
{\it
\vspace{-0.55cm}
\begin{itemize}
\renewcommand{\labelitemi}{{\textcolor{dark}{$\ast$}}}
\item Reading: Section 5.3 of OpenIntro Statistics
%\item Videos: To be posted
\item Test yourself:
\begin{enumerate}
\item Describe how the two sample means test is different from the paired means test, both conceptually and in terms of the calculation of the standard error.
\item A 95\% confidence interval for the difference between the number of calories consumed by mature and juvenile cats ($\mu_{mat} - \mu_{juv}$) is (80 calories, 100 calories). Interpret this interval, and determine if it suggests a significant difference between the two means.
\end{enumerate}
\end{itemize}
}}

%

\vspace{0.48cm}

%

\begin{enumerate}[resume]
\renewcommand\labelenumi{\textcolor{light}{\textbf{LO \theenumi.}}}

\item Calculate the power of a test for a given effect size and significance level in two steps: (1) Find the cutoff for the sample statistic that will allow the null hypothesis to be rejected at the given significance level, (2) Calculate the probability of obtaining that sample statistic given the effect size.

\item Explain how power changes for changes in effect size, sample size, significance level, and standard error.

\end{enumerate}

%

\vspace{0.48cm}


\begin{enumerate}[resume]
\renewcommand\labelenumi{\textcolor{light}{\textbf{LO \theenumi.}}}

\item Define analysis of variance (ANOVA) as a statistical inference method that is used to determine if the variability in the sample means is so large that it seems unlikely to be from chance alone by simultaneously considering many groups at once.

\item Recognize that the null hypothesis in ANOVA sets all means equal to each other, and the alternative hypothesis suggest that at least one mean is different.
\begin{itemize}
\item[] $H_0: \mu_1 = \mu_2 = \cdots = \mu_k$
\item[] $H_A:$ At least one mean is different
\end{itemize} 

\item List the conditions necessary for performing ANOVA
\begin{enumerate}
\item[(1)] the observations should be independent within and across groups
\item[(2)] the data within each group are nearly normal
\item[(3)] the variability across the groups is about equal
\end{enumerate}
and check if they are met using graphical diagnostics.

\item Recognize that the test statistic for ANOVA, the F statistic, is calculated as the ratio of the mean square between groups (MSG, variability between groups) and mean square error (MSE, variability within errors), and has two degrees of freedom, one for the numerator ($df_{G} = k - 1$, where $k$ is the number of groups) and one for the denominator ($df_{E} = n - k$, where $n$ is the total sample size).
\begin{itemize}
\item[-] Note that you won't be expected to calculate MSG or MSE from the raw data, but you should have a conceptual understanding of how they're calculated and what they measure.
\end{itemize} 

\item Describe why calculation of the p-value for ANOVA is always ``one sided".

\item Describe why conducting many $t$-tests for differences between each pair of means leads to an increased Type 1 Error rate, and we use a corrected significance level (Bonferroni corection, $\alpha^\star = \alpha / K$, where $K$ is the e number of comparisons being considered) to combat inflating this error rate.

\item Describe why it is possible to reject the null hypothesis in ANOVA but not find significant differences between groups as a result of pairwise comparisons.

\end{enumerate}

\gray{
{\it
\vspace{-0.55cm}
\begin{itemize}
\renewcommand{\labelitemi}{{\textcolor{dark}{$\ast$}}}
\item Reading: Section 5.5 of OpenIntro Statistics
%\item Videos: To be posted
\item Test yourself:
\begin{enumerate}
\item We would like to compare the average income of Americans who live in the Northeast, Midwest, South, and West. What are the appropriate hypotheses?
\item Suppose the sample in the question above has 1000 observations, what are the degrees of freedom associated with the F-statistic?
\item Suppose the null hypothesis is rejected. Describe how we would discover which regions' averages are different from each other? Make sure to discuss how many pairwise comparisons we would need to make, and what the corrected significance level would be.
\item What visualizations are useful for checking each of the conditions required for performing ANOVA?
\end{enumerate}
\end{itemize}
}}


\end{document}