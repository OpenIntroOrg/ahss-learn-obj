\documentclass[11pt]{article}
\usepackage[top=1.5cm,bottom=2cm,left=2cm,right= 2cm]{geometry}
\geometry{letterpaper}                   % ... or a4paper or a5paper or ... 
%\geometry{landscape}                % Activate for for rotated page geometry
\usepackage[parfill]{parskip}    % Activate to begin paragraphs with an empty line rather than an indent
\usepackage{graphicx}
\usepackage{amssymb}
\usepackage{epstopdf}
\usepackage{amsmath}            
\usepackage{multirow}    
\usepackage{multicol}    
\usepackage{changepage}
\usepackage{lscape}
\usepackage{enumitem}
\usepackage{ulem}
\DeclareGraphicsRule{.tif}{png}{.png}{`convert #1 `dirname #1`/`basename #1 .tif`.png}

\usepackage{xcolor}

\definecolor{oiB}{rgb}{.337,.608,.741}
\definecolor{oiR}{rgb}{.941,.318,.200}
\definecolor{oiG}{rgb}{.298,.447,.114}
\definecolor{oiY}{rgb}{.957,.863,0}

\definecolor{light}{rgb}{.337,.608,.741}
\definecolor{dark}{rgb}{.337,.608,.741}

\usepackage[colorlinks=false,pdfborder={0 0 0},urlcolor= dark,colorlinks=true,linkcolor=black]{hyperref}

\newcommand{\light}[1]{\textcolor{light}{\textbf{#1}}}
\newcommand{\dark}[1]{\textcolor{dark}{#1}}
\newcommand{\gray}[1]{\textcolor{gray}{#1}}


%\date{}                                           % Activate to display a given date or no date

%

\begin{document}

{\LARGE \textcolor{oiB}{Learning Objectives \hfill Chapter 7: Inference for numerical data}} \\

%

\begin{enumerate}
\renewcommand\labelenumi{\textcolor{light}{\textbf{LO \theenumi.}}}

\item Use the $t$-distribution for inference on a single mean, mean of paired difference, and difference of independent means.

\item Explain why the $t$-distribution helps make up for the additional variability introduced by using $s$ (sample standard deviation) in calculation of the standard error, in place of $\sigma$ (population standard deviation).

\item Describe how the $t$-distribution is different from the normal distribution, and what �heavy tail� means in this context. 

\item Note that the $t$-distribution has a single parameter, degrees of freedom, and as the degrees of freedom increases this distribution approaches the normal distribution. 

\item Calculate the required sample size to obtain a given margin of error at a given confidence level by working backwards from the given margin of error.  Use $\sigma$ if it is given, along with the Z-score that corresponds to the confidence level.

\item Use a $t$-statistic, with degrees of freedom $df = n - 1$ for inference for a population mean:
\begin{itemize}
\item[-] Standard error: $SE = \frac{s}{\sqrt{n}}$
\item[-] Confidence interval: $\bar{x} \pm t_{df}^\star SE$
\item[-] Hypothesis test: $T_{df} = \frac{\bar{x} - \text{null value}}{SE}$
\end{itemize}

\item Describe how to obtain a p-value for a $t$-test and a critical $t$-score ($t^\star_{df}$) for a confidence interval.

\end{enumerate}

\gray{
{\it
\vspace{-0.55cm}
\begin{itemize}
\renewcommand{\labelitemi}{{\textcolor{dark}{$\ast$}}}
\item Reading: Section 7.1 of Advanced High School Statistics
%\item Videos: To be posted
\item Test yourself:
\begin{enumerate}
\item What is the $t^\star$ for a 95\% confidence interval for a mean, where the sample size is 13.
\item In a random sample of 1,017 Americans 60\% said they do not trust the mass media when it comes to reporting the news fully, accurately, and fairly. The standard error associated with this estimate is 0.015 (1.5\%). What is the margin of error a 95\% confidence level? Calculate a 95\% confidence interval and interpret it in context. 
\item What is the p-value for a hypothesis test where the alternative hypothesis is two-sided, the sample size is 20, and the test statistic, T, is calculated to be 1.75? 
\end{enumerate}
\end{itemize}
}}

%

\vspace{0.48cm}

%

\begin{enumerate}[resume]
\renewcommand\labelenumi{\textcolor{light}{\textbf{LO \theenumi.}}}

\item Define observations as paired if each observation in one dataset has a special correspondence or connection with exactly one observation in the other data set. 

\item Carry out inference for paired data by first subtracting the paired observations from each other, and then treating the set of differences as a new numerical variable on which to do inference (such as a confidence interval or hypothesis test for the average difference). 

\item Calculate the standard error of the difference between two paired (dependent) samples as \\$SE = \frac{s_{diff}}{\sqrt{n_{diff}}}$ and use this standard error in hypothesis testing and confidence intervals for the mean of paired differences. 

\item Use a $t$-statistic, with degrees of freedom $df = n_{diff} - 1$ for inference for a population mean:
\begin{itemize}
\item[-] Standard error: $SE = \frac{s}{\sqrt{n}}$
\item[-] Confidence interval: $\bar{x}_{diff} \pm t_{df}^\star SE$
\item[-] Hypothesis test: $T_{df} = \frac{\bar{x}_{diff} - \text{null value}}{SE}$. Note that null value is often 0, since often $H_0: \mu_{diff} = 0$.
\end{itemize}


\end{enumerate}

\gray{
{\it
\vspace{-0.55cm}
\begin{itemize}
\renewcommand{\labelitemi}{{\textcolor{dark}{$\ast$}}}
\item Reading: Section 7.2 of Advanced High School Statistics
%\item Videos: To be posted
\item Test yourself:
\begin{enumerate}
\item 20 cardiac patients' blood pressure is measured before taking a medication, and after. For a given patient, are the before and after blood pressure measurements dependent (paired) or independent?
\item A random sample of 100 students were obtained and then randomly assigned into two equal sized groups. One group went on a roller coaster while the other in a simulator at an amusement park. Afterwards their blood pressure measurements were taken. Are the measurements dependent (paired) or independent?
\end{enumerate}
\end{itemize}
}}

%

\vspace{0.48cm}

%

\begin{enumerate}[resume]
\renewcommand\labelenumi{\textcolor{light}{\textbf{LO \theenumi.}}}

\item Calculate the standard error of the difference between means of two independent samples as \\$SE = \sqrt{\frac{s_1^2}{n_1} + \frac{s_2^2}{n_2}}$, and use this standard error in hypothesis testing and confidence intervals comparing means of independent groups.

\item Use a $t$-statistic, with degrees of freedom $df = min(n_1 - 1, n_2 - 1)$ or as provided by calculator for inference for the difference between two means:
\begin{itemize}
\item[-] Standard error: $SE = \sqrt{\frac{s_1^2}{n_1} + \frac{s_2^2}{n_2}}$
\item[-] Confidence interval: $(\bar{x}_1 - \bar{x}_2) \pm t_{df}^\star SE$
\item[-] Hypothesis test: $T_{df} = \frac{(\bar{x}_1 - \bar{x}_2) - \text{null diff}}{SE}$. Note that null diff is often 0, since often $H_0: \mu_1 - \mu_2 = 0$.
\end{itemize}

\item Recognize that a good interpretation of a confidence interval for the difference of means between two parameters includes a comparative statement (mentioning which group has the larger parameter). 

\item Recognize that a confidence interval for the difference between two parameters that doesn�t include 0 is in agreement with a hypothesis test where the null hypothesis (that the two parameters equal each other) is rejected. 

\end{enumerate}

\gray{
{\it
\vspace{-0.55cm}
\begin{itemize}
\renewcommand{\labelitemi}{{\textcolor{dark}{$\ast$}}}
\item Reading: Section 7.3 of Advanced High School Statistics
%\item Videos: To be posted
\item Test yourself:
\begin{enumerate}
\item Describe how the two sample means test is different from the paired means test, both conceptually and in terms of the calculation of the standard error.
\item A 95\% confidence interval for the difference between the number of calories consumed by mature and juvenile cats ($\mu_{mat} - \mu_{juv}$) is (80 calories, 100 calories). Interpret this interval, and determine if it suggests a significant difference between the two means.
\end{enumerate}
\end{itemize}
}}






\end{document}